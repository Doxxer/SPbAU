\documentclass[10pt,a4paper]{article} 
\usepackage[T1,T2A]{fontenc} 
\usepackage[utf8]{inputenc} 
\usepackage[russian]{babel} 
\usepackage{amssymb} 
\usepackage{amsmath} 
\usepackage{amsthm} 
\usepackage{latexsym} 
\usepackage{enumerate} 
\usepackage[cm]{fullpage} 
\usepackage[pdftex]{graphicx}
\sloppy

\begin{document}

\title{Графы. HW\#3} 
\author{Тураев Тимур, 504 (SE)} 
\maketitle 
\begin{enumerate}
	\item[1.17.] \textit{Доказать, что либо у графа, либо у его дополнения диаметр не больше 3. В частности, один из графов обязательно связен}.\\
	Докажем, что, если дан граф $G$, в котором $diam(G) \geqslant 3$, то $diam(\overline{G}) \leqslant 3$.\\
	Пусть $u$ и $v$ - две вершины в графе $G$, расстояние между которыми не меньше 3. Выберем произвольно две вершины $x$ и $y$ в графе $G$ и докажем, что расстоние между ними в графе-дополнении будет не больше 3.\\
	Заметим, что вершина $x$ не смежна хотя бы с одной из вершин $u$ или $v$ (в противном случае получим, что у нас есть путь $uxv$ длиной 2). Без потери общности можем считать, что вершина $x$ не смежна с $u$.
	\begin{enumerate}
	\item[1.] Пусть вершина $y$ не смежна с $u$, тогда в графе-дополнении все три вершины становятся смежными и у нас появляется путь $xuy$ длиной 2.
	\item[2.] Пусть вершина $y$ смежна с $u$, тогда она не смежна с $v$ (по тем же соображениям). Тогда в графе-дополнении появляется путь $xuvy$ длиной 3. (так как вершины $u$ и $v$ были не смежны в графе $G$, значит будут смежны в графе $\overline{G}$)	
	\end{enumerate}
	Таким образом, любые две вершины $x$ и $y$ будут находиться друг от друга на расстоянии не больше 3 в графе $\overline{G}$ - он и будет связным.
	
	\item[3.15.] \textit{Пусть для некоторой пары несмежных вершин $x$ и $y$ в простом графе $G$ выполняется неравенство $deg(x)+deg(y) \geqslant n = |V(G)| > 2$. Доказать, что в графе $G$ существует гамильтонов цикл тогда и только тогда, когда он существует в графе $G+\{x,y\}$.}
	\begin{enumerate}
		\item[$\Rightarrow$] Очевидно, если в графе $G$ был цикл, то добавление ребра в в граф не изменит этот цикл.
		
		\item[$\Leftarrow$] Пусть в графе $G+\{x,y\}$ имеется гамильтонов цикл. Предположим, он не проходит по ребру $\{x,y\}$, тогда, удалив это ребро, мы сохраним цикл.\\
		Пусть этот цикл проходит по ребру $\{x,y\}$. Удалив это ребро, получим гамильтонов путь из вершины $x$ в вершину $y$. Применим лемму 3.8: она говорит о том, что если есть гамильтонов путь между двумя несмежными вершинами в графе, то достаточным условием существования гамильтонова цикла в этом графе является выполнение неравенства $deg(x)+deg(y) \geqslant n = |V(G)| > 2$, что нам и дано.
	\end{enumerate}	
	
	\item[4.3.] \textit{Подмножество $U$ множества вершин графа называется независимым, если никакие две вершины этого подмножества не являются смежными. Обозначим через $\alpha(G)$ количество вершин в максимальном независимом подмножестве графа $G$. Доказать, что для любого графа $G$, построенного на $n$ врешинах, справедливо неравенство $\chi(G) \cdot \alpha(G) \geqslant n, n=|V(G)|$ }\\
	Все вершины графа $G$ могут быть разбиты на $\chi(G)$ монохроматических классов. Каждый класс, очевидно, представляет собой независимое множество вершин, а значит его размер не превышает $\alpha(G)$. Очевидно, это выполняется и для максимального по мощности монохроматического класса. Его размер не меньше чем $\dfrac{|V(G)|}{\chi(G)}$ (это очевидно следует из противоречивости противоположного утверждения: в этом случае получится, что общее число вершин $|V(G)| < |V(G)|$). Отсюда: $\alpha(G) \geqslant \dfrac{|V(G)|}{\chi(G)}$ или $\chi(G) \cdot \alpha(G) \geqslant |V(G)|$.
	
	\item[4.5.] \textit{Доказать, что для любого простого графа $G$ на $n$ вершинах $\chi(G) + \chi(\overline{G}) \leqslant n+1$ и $\chi(G) \cdot \chi(\overline{G}) \geqslant n$}
	
	\begin{enumerate}
		\item[1.] Докажем по индукции. База: $n=1$. Очевидно, что $\chi(G) =\chi(\overline{G}) = 1$\\
		Предположение индукции: выберем в графе какую-нибудь вершину $v$. Тогда $\chi(G-v) + \chi(\overline{G-v}) \leqslant n$\\
		Индукционный переход. Заметим, что $\chi(G) \leqslant \chi(G-v) + 1$ и $\chi(\overline{G}) \leqslant \chi(\overline{G-v}) + 1$. Это очевидно: при добавлении в граф новой вершины, число цветов, необходимых для покраски не может увеличиться больше чем на 1.\\
		Рассмотрим степень вершины $v$. Если $deg(v) < \chi(G-v)$, то новый цвет нам и не понадобится, а поэтому $\chi(G) = \chi(G-v)$, следовательно $\chi(G) + \chi(\overline{G}) \leqslant \chi(G-v) + \chi(\overline{G-v}) + 1 \leqslant n + 1$\\
		Аналогичные рассуждения можно провести и для дополнения $G$: если $deg(v) > (n-1) - \chi(\overline{G-v})$, то $\chi(\overline{G}) = \chi(\overline{G-v})$, следовательно $\chi(G) + \chi(\overline{G}) \leqslant \chi(G-v) + 1 + \chi(\overline{G-v}) \leqslant n + 1$
		Остался случай, когда $deg(v) \geqslant \chi(G-v)$ и $deg(v) \leqslant (n-1) - \chi(\overline{G-v})$: но тогда выделив из неравенств хроматические числа получим, что $\chi(G-v) + \chi(\overline{G-v}) \leqslant n-1$, следовательно $\chi(G) + \chi(\overline{G}) \leqslant \chi(G-v) + 1 + \chi(\overline{G-v}) + 1 \leqslant n - 1+ 2 = n+1$
		\item[2.] $\chi(\overline{G}) \geqslant \alpha(G)$, так как наибольшей антиклике соответствует клика в графе-дополнении, а значит, для ее покраски нужно не меньше $\alpha(G)$ цветов. Применяем результат задачи 4.3, получаем, что $\chi(G) \cdot \chi(\overline{G}) \geqslant n$
	\end{enumerate}
	
	\item[4.8.] \textit{Доказать, что если в ориентированном графе нет путей длины, большей, чем $m$ (в частности, нет циклов), то $\chi(G) \leqslant m+1$}\\
	Докажем, что граф, в котором нет пути длины большей чем $m$, можно корректно покрасить в не более чем $m+1$ цвет. Это будет означать, что хроматическое число этого графа не превышает $m+1$.\\
	Алгоритм: покрасим каждую вершину в цвет, равный длине максимального пути, который заканчивается в этой вершине. Очевидно, что все истоки получат цвет ноль. Рассмотрим какое-нибудь ориентированное ребро $(x, y)$: легко понять, что оно соединяет вершины, значения цветов которых строго возрастают -- действительно, если длина максимального пути, который заканчивается в вершине $x$ равна, скажем, $A$, то в вершине $y$ эта длина не может быть меньше $A$ и не может быть равна $A$ - так как всегда существует путь длиной по крайней мере $A+1$. (к слову, длина в вершине $y$ не обязательно равна $A+1$ -- в этой вершине может заканчиваться какой-либо другой очень длинный путь). Итак, цвет конца ребра всегда больше цвета его начала -- значит, такая раскраска всегда корректна. \\
	Так как по условию в графе $G$ нет путей длины больше, чем $m$, то длина максимального пути в этом графе $k \leqslant m$, значит этот путь состоит ровно из $k+1$ вершины, причем первая является истоком (в противном случае существовал бы путь длины больше, чем $k$), а это значит, что максимальный цвет, который может быть использован алгоритмом - $k$. Значит, всего цветов мы используем не больше $k+1$ (не забудем про нулевой цвет). Отсюда заключаем, что $\chi(G) \leqslant k+1 \leqslant m+1$.
	
	\item[4.9.] \textit{Обозначим через $K_{n, n}^{*}$ граф, полученный из полного двудольного графа $K_{n, n}$ с блоками $X = \{x_1, \ldots, x_n\}$ и $Y = \{y_1, \ldots, y_n\}$ удалением ребер ${x_i, y_i}$. Предположим, что граф $K_{n, n}^{*}$ окрашивается с помощью описанного в пункте 2 жадного алгоритма. Предъявить два способа начального упорядочивания вершин этого графа, для одного из которых алгоритм окрасит вершины в два цвета, а для второго — в $n$ цветов.}
	\begin{enumerate}
		\item[1.] В два цвета: $\{x_1, \ldots, x_n, y_1, \ldots, y_n\}$. Алгоритм сначала покрасит весь блок $X$ в цвет 1, т.к. каждая следующая вершина в этом блоке будет несмежна ни с какой из предыдущих, а поэтому всем проставится минимальный из возможных цветов -- 1. Затем, проходя по блоку $Y$ алгоритм проставит каждой вершине цвет 2, так как у каждой вершины будет ровно $n-1$ сосед с меньшим номером и цветом 1, а значит минимальный неиспользуемый - это 2.
		\item[2.] В $n$ цветов: $\{x_1, y_1, x_2, y_2, \ldots, x_n, y_n\}$. Рассуждения похожие: сначала вершине $x_1$ проставится цвет 1, равно как и вершине $y_1$  (поскольку они несмежны). Вершине $x_2$ -- цвет 2, так как у нее будет ровно одна смежная вершина с меньшим номером, которая покрашена в цвет 1. Аналогично вершина $y_2$ получит цвет 2. Вообще, вершина $x_i$ будет покрашена в цвет $i$, поскольку у нее будет ровно $i-1$ сосед с меньшим номером (все лежат в противоположном блоке) и у каждого из них цвет от 1 до $i-1$, поэтому вершина $x_i$ получит цвет  $i$.
		Значит, весь граф будет покрашен в $n$ цветов.
	\end{enumerate}
	
	\item[4.10.] \textit{Доказать, что в любом графе $G$ существует такое линейное упорядочение его вершин, при котором жадный алгоритм раскраски окрасит вершины графа ровно в $\chi(G)$ цветов.}\\
	Такой обход действительно существует. Представим, что граф $G$ уже покрашен в $\chi(G)$ цветов. Корректно перекрасим все вершины, покрашенные не в первый цвет -- в первый. Если этого сделать больше нельзя, то это значит, что каждая вершина имеет в соседях вершину первого цвета (иначе мы бы смогли покрасить ее в первый цвет, ничего не сломав). Дальше, перекрасим все вершины, которые имею цвет не меньше второго - во второй. И так далее. Число цветов уменьшиться не может, по причине покраски в $\chi(G)$ цветов. Больше тоже быть не может: мы не вводим новых цветов.\\
	Теперь ясно в каком порядке нужно отсортировать вершины так, чтобы жадный алгоритм покрасил этот граф ровно в $\chi(G)$ цветов: вначале выдадем все вершины цвета 1 (виртуального), затем - 2 и так далее.\\
	Алгоритм пойдет по вершинам и поставит им цвет 1, так как все они несмежны друг с другом (образуют антиклику), затем всем вершинам, которые были "покрашены" (виртуально) во второй цвет - корректно окрасятся во второй, т.к. во-первых все они не смежны, а также у каждой есть в соседях вершины уже покрашенные в первый цвет. И так далее. Очевидно, что алгоритм окрасит весь граф в $\chi(G)$ цветов.
	
	\item[4.17.] \textit{Пусть $G_2 = K_2$ , а графы $G_k, k > 2$, получаются из графов $G_{k-1}$ с помощью описанной в теореме 4.17 процедуры. Подсчитать количество вершин в графе $G_k$.}\\
	Согласно теореме 4.17, на каждом шаге число вершин удваивается и увеличивается на единицу. Значит, рекуррентное соотношение выглядит так: $|G_{n+1}| = 2 \cdot |G_n| + 1$, $|G_2| = 2$. Линейное неоднородное рекуррентное соотношение первого порядка мы решать умеем (делали на практике по комбинаторике). Введем новую переменную $\tilde{a}_n = a_n + c$:
	\[ \tilde{a}_{n+1} - c = 2 \cdot \tilde{a_n} - 2c  + 1 \]
	\[ \tilde{a}_{n+1} = 2 \cdot \tilde{a_n} - c  + 1 \]
	Выберем $c = 1$ (т.е. $\tilde{a_2} = a_2 +c = 2 + 1 = 3$), чтобы получить однородное рекуррентное соотношение. Его решением будет:
	\[ \tilde{a}_{n} = 2^{n-2} \cdot \tilde{a_2}, \tilde{a_2} = 2\]
	Значит решением исходного неоднородного рекуррентного соотношения будет
	\[ |G_n| = 3 \cdot 2^{n-2} - 1, n \in [2, + \infty) \]
\end{enumerate}
\end{document} 
