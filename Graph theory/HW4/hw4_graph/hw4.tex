\documentclass[russian]{article}
\usepackage[T1]{fontenc}
\usepackage[utf8]{inputenc}
\usepackage{geometry}
\geometry{verbose,tmargin=2cm,bmargin=2cm,lmargin=2cm,rmargin=2cm}
\usepackage{float}
\usepackage{textcomp}
\usepackage{amssymb}
\usepackage{graphicx}

\makeatletter

%%%%%%%%%%%%%%%%%%%%%%%%%%%%%% LyX specific LaTeX commands.
\DeclareRobustCommand{\cyrtext}{%
  \fontencoding{T2A}\selectfont\def\encodingdefault{T2A}}
\DeclareRobustCommand{\textcyr}[1]{\leavevmode{\cyrtext #1}}
\AtBeginDocument{\DeclareFontEncoding{T2A}{}{}}

%% A simple dot to overcome graphicx limitations
\newcommand{\lyxdot}{.}


\@ifundefined{date}{}{\date{}}
%%%%%%%%%%%%%%%%%%%%%%%%%%%%%% User specified LaTeX commands.
\usepackage[T2A]{fontenc}

\makeatother

\usepackage{babel}
\begin{document}

\title{Графы. HW\#4}


\author{Тураев Тимур, 504 (SE)}

\maketitle

\paragraph{1.4}

Проведем 9 горизонтальных и 3 вертикальные прямые. У нас образуется
9 горизонтальных троек. Покрасить три точки в два цвета можно $2^{3}=8$
различными способами. Следовательно, по принципу Дирихле среди 9 троек
найдутся хотя бы две, покрашенные одинаково.

Рассмотрим эти две одинаково покрашенные тройки точек -- таким образом,
две горизонтальные прямые мы выбрали. Рассмотрим одну из этих троек.
Опять же по принципу Дирихле среди трех точек найдутся хотя бы две,
покрашенные одинаково -- без потери общности, пусть эти две точки
лежат на вертикальных прямых с номерами 1 и 2. Но поскольку обе тройки
покрашены одинаково, то и во второй тройке на вертикальных прямых
с номерами 1 и 2 будут лежать две точки покрашенные в тот же цвет,
что и в первой тройке. Что и требовалось доказать.


\paragraph{5.5}

Возьмем 4 точки $A,B,C,D$ как показано на рисунке.

\begin{center}
\includegraphics[scale=0.5]{5\lyxdot 5}
\par\end{center}

При этом пары точек $AB=AC=BC=BC=CD=1$ -- расположены на расстоянии
1. По принципу Дирихле хотя бы две из этих точек покрашены в одинаковый
цвет. Если эти две точки образуют одну из перечисленных пар, то задача
решена. В противном случае точки $A$ и $D$ должны быть покрашены
в один цвет, например, в цвет 1. Тогда {}``повернем'' ромб $ABDC$
относительно точки $A$ так, что в полученном ромбе $AB'D'C'$ расстояние
$DD'=1$. Рассуждая аналогично придем к выводу, что либо задача решена,
либо точки $A$ и $D'$ покрашены в цвет 1. Следовательно, точки $D$
и $D'$ покрашены в один цвет. Однако по построению расстояние между
ними равно 1 -- и задача решена.


\paragraph{5.6}

$R(3,4)$-- значит надо гарантированно найти либо красную 3-клику,
либо синюю 4-клику. Покажем, как это сделать в полном графе на 9 вершинах.

Рассмотрим произвольную вершину графа, например с номером 1. Она связана
восемью ребрами со всеми остальными.
\begin{itemize}
\item Пусть 4 из них покрашены в красный цвет, а 4 в синий. Если мы концы
хотя бы двух ребер соединим красным ребром, то получим красный треугольник
(красную 3-клику). Если же все концы соединим синими ребрами, то получим
синюю 4-клику:
\end{itemize}
\begin{center}
\includegraphics[scale=0.5]{5\lyxdot 6_1}
\par\end{center}
\begin{itemize}
\item В случае если красных ребер больше 4 ситуация ровно та же: достаточно
зафиксировать любые четыре красные ребра и свести задачу к предыдущей.
\item Пусть красных ребер не больше двух, следовательно, синих 6 или более.
Пусть концы 6 синих ребер -- вершины с номерами $2,3,4,5,6,7$. Они
в графе образуют полную 6-клику. Как мы знаем, в полном графе на 6
вершинах найдется одноцветный треугольник. Если это красный треугольник
-- это нам подходит. Пусть это синий треугольник на вершинах с номерами
$2,3,4$. Вспомним, что вершина 1 соединена с каждой из них синими
ребрами -- получаем синюю 4-клику.
\item Наконец пусть из вершины выходят 3 красных ребра и 5 синих. Рассмотрим
все остальные восемь вершин. Если хотя бы для одной из них соотношение
ребер другое, то, как мы уже доказали, в этом случае задача будет
решена. Тогда остается рассмотреть, что будет если изо всех вершин
выходит по 3 красных и 5 синих ребра. Каждая вершина является началом
и концом трех красных ребер. Всего вершин 9. Следовательно, всего
красных ребер в графе: $\frac{3\cdot9}{2}=\frac{27}{2}=13,5$. Но
нецелого числа ребер быть не может, следовательно и такая раскраска
в графе невозможна.
\end{itemize}
Таким образом, мы рассмотрели все возможные раскраски и показали,
что в полном графе на 9 вершинах всегда найдется либо красная 3-клика,
либо синяя 4-клика.

Покажем теперь, что для полного графа на 8 вершинах условие выполнено
не будет. Приведем контрпример:

\begin{center}
\includegraphics[scale=0.5]{5\lyxdot 6_2}
\par\end{center}

По построению красных 3-клик быть не может. Так как граф симметричен,
рассмотрим лишь одну ее вершину и 4 исходящих из нее синих ребра.
Видно, что и синюю 4-клику мы построить не сможем - всегда будет мешать
красное ребро.

Таким образом, $R(3,4)=9$. Что и требовалось доказать.


\paragraph{5.7}

$R(3,5)$ -- значит надо гарантированно найти либо красную 3-клику,
либо синюю 5-клику. Воспользуемся теоремой Эрдеша-Секереша, что для
любых $p,q\geqslant3$ справедливо 
\[
R(p,q)\leqslant R(p-1,q)+R(p,q-1)
\]
В нашем случае:
\[
R(3,5)\leqslant R(2,5)+R(3,4)=5+9=14
\]


Покажем что для графа на 13 вершинах можно привести такую раскраску,
что в графе нельзя будет найти красную 3-клику или синюю 5-клику.

Раскрасим некоторые ребра графа в красный цвет, как показано на рисунке:

\begin{center}
\includegraphics[scale=0.5]{5\lyxdot 7_1}
\par\end{center}

Видно что красных треугольников нет. Будем пытаться построить синюю
5-клику. Выберем какую-либо вершину (на рисунке выше она закрашена
синим). Сразу заметим, что незакрашенные вершины включать в синюю
клику нельзя, поскольку он соединены с синей вершиной красными ребрами.

Покажем, что не включать какую-либо из фиолетовых вершин нельзя. Действительно,
попробуем построить пятиугольник, избегая красных ребер. Единственный
вариант: брать вершины через одну, т.е. все зеленые, но и это в итоге
приведет даст красное ребро:

\begin{center}
\includegraphics[scale=0.5]{5\lyxdot 7_2}
\par\end{center}

Значит, фиолетовая быть должна. В силу симметрии можно выбрать любую
-- выберем левую.

Попробуем соединить синюю вершину с зеленой -- получим в итоге красное
ребро:

\begin{center}
\includegraphics[scale=0.5]{5\lyxdot 7_3}
\par\end{center}

Теперь попробуем последний вариант: соединить синюю с оранжевой. Но
и здесь тоже получаем красное ребро:

\begin{center}
\includegraphics[scale=0.5]{5\lyxdot 7_4}
\par\end{center}

Таким образом, для графа на 13 вершинах можно привести контрпример.
Следовательно, $R(3,5)=14$.


\paragraph{5.8}

Поскольку в задаче фигурирует три числа, которые должны быть покрашены
в один цвет, задачу хотелось свести к нахождению одноцветного треугольника,
что связано с числами Рамсея. При этом стороны треугольника должны
как-то коррелировать с условием $a+b=c$. После многочисленных попыток
правильно расположить числа на ребрах или на вершинах, чтобы всегда
нашлась нужная покраска, в итоге получилось следующее.

Рассмотрим на примере. Возьмем $k=2$, а $N=R(3,3)=6$. Покрасим числа
1, 3, 6 в синий цвет, а 2, 4, 5 в красный. Возьмем полный граф на
6 вершины. И покрасим вершины соответственно их номерам, т.е. вершины
с номерами 1, 3, 6 в синий цвет, а 2, 4, 5 в красный. Далее будем
красить все ребра следующим образом. Ребро $(u,v)$ покрасим в тот
цвет, в который покрашена вершина с номером $|u-v|$:

\begin{center}
\includegraphics[scale=0.5]{5\lyxdot 8}
\par\end{center}

В этом графе гарантированно найдется одноцветный треугольник. Например,
2\_4\_6. Это означает по построению, что вершины с номерами $4-2=2,6-4=2,6-2=4$
покрашены в один цвет. Соответственно, положив $a=2,b=2,c=4$, получим
$a+b=c$, что нам и требуется.

Ровно то же справедливо и для любых других $k$. Здесь надо брать
$N=R(3,\dots,3)$ (число троек равно $k$) и проводить аналогичные
рассуждения.

Остается показать, что такое $N=R(3,\dots,3)$ всегда существует.
Но об этом говорится в теореме Рамсея. Следовательно, наше доказательство
закончено.
\end{document}
