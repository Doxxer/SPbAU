\documentclass[10pt,a4paper]{article}
\usepackage[T1,T2A]{fontenc}
\usepackage[utf8]{inputenc}
\usepackage[russian]{babel}
\usepackage{amssymb}
\usepackage{amsmath}
\usepackage{amsthm}
\usepackage{latexsym}
\usepackage{enumerate}
\usepackage[cm]{fullpage}

\begin{document}

\title{Графы. HW\#1}
\author{Тураев Тимур, 504 (SE)}
\maketitle

\begin{enumerate}
	\renewcommand\labelenumi{\bfseries\theenumi}
	\item \textit{Пусть матрица $A$ есть матрица смежности графа $G$. Показать, что элемент $a_{i,j}^k$ $k$-ой степени матрицы $A$ определяет количество путей длины $k$ из вершины $i$  в вершину $j$.}\\
	Очевидно, что матрица смежности представляет собой ответ на задачу при $k=1$ ~-- она содержит количество путей длины $1$ между каждой парой вершин. Пусть для некоторого $k$ ответ найден ~-- обозначим через $d_k$ его матрицу. Тогда очевидно, что
	\[
	d_{k+1}[i][j]=\sum\limits_{p=1}^n d_k[i][p] \cdot A[p][j],
	\]так как мы считаем число путей добраться до каждой вершины, а потом от этой каждой - до нужной нам.\\
	Но это есть не что иное, как определение произведения матриц $d_k$ и $A$. Значит, элемент $a_{i,j}^k$ матрицы $A^k$ есть число путей длины $k$ из вершины $i$ в вершину $j$.	
	\item \textit{Доказать, что в определении связанности двух вершин можно вместо понятия пути использовать как понятие маршрута, так и понятие простого пути.}
	
	\begin{enumerate}
	\item Докажем, что если между двумя различными вершинами существует путь, то существует и простой путь (обратное утверждение тривиально).
	Предположим, что в пути существует повторяющаяся вершина:
	\[ x_0,x_1,\ldots x_k,y_0,y_1,\ldots,y_p,x_k,x_{k+1},\ldots,x_n\]
	Тогда можно удалить из пути подпуть $y_0,y_1,\ldots,y_p,x_k$ и получить путь
	\[ x_0,x_1,\ldots x_k,x_{k+1},\ldots,x_n,\] в котором уже нет повторяющейся вершины $x_k$. Если в пути еще есть повторяющиеся вершины ~-- делаем еще шаг алгоритма. В итоге получится путь из различных вершин, т.е. простой путь.
	\item Доказательство того, что если между двумя различными вершинами существует маршртут, то существует и простой путь, аналогично предыдущему - мы высекаем из марштура подмаршрут, который проходит по одному и тому же ребру.
	\end{enumerate}		
	
	\item \textit{Пусть в графе $G$ ровно две вершины имеют нечетную степень. Доказать, что они являются связанными.}\\
	Предположим обратное. Тогда, граф $G$ состоит как минимум из двух компонент связности. Рассмотрим ту компоненту связности, где находится первая вершина. В этом подграфе всего 1 вершина нечетной степени, остальные четной. Получили противоречие с леммой о рукопожатиях ~-- в любом конечном графе четное число вершин нечетной степени.\\
	  Значит, наше предположение о несвязности неверно и эти две вершины связаны.
	\item \textit{Доказать или опровергнуть следующее утверждение: объединение двух различных маршрутов, соединяющих две вершины, содержит простой цикл.}\\
	Это неверное утверждение. Пусть у нас есть граф из двух вершин $x_0$ и $x_1$, соединенных ребром $e$. Рассмотрим такие маршруты:\\
	маршрут1: $x_0,e,x_1$\\
	маршрут2: $x_0,e,x_1,e,x_0$\\
 	Объединение маршрутов даст маршрут 2, в котором, очевидно, никакого цикла нет.
	\item \textit{Доказать или опровергнуть следующее утверждение: объединение двух различных простых путей, соединяющих две вершины, содержит простой цикл.}\\
	Пусть есть две вершины $a$ и $b$ и 2 различных простых пути между ними.
	Так как данные простые пути различны, то найдется такая вершина $x_k$ (возможно, начальная, $a$), на которой будет "развилка". Аналогично строится догадка, что среди оставшихся вершин найдется такая (возможно, конечная, $b$) вершина, где пути сойдутся - $y_k$. Рассмотрим следующий путь: от $x_k$ до $y_k$ по первому простому пути, затем обратно к $x_k$ по второму. Получили замкнутый путь, т.е. цикл. Получение простого цикла (цикла, в котором повторяются начальные и конечные вершины) из цилка очень простое - удалим из цикла какое-нибудь ребро - получим путь. Применим к этому пути задачу 2. Получим простой путь. Восстановим ребро - получим, очевидно, простой цикл.
	\item \textit{Доказать, что в простом графе, содержащем $m\geqslant(n-1)$ ребер, существует как минимум $m-n+1$ циклов.}\\
	ВОПРОС. В простом графе или связном? Для связного графа доказательство тривиальное - граф не дерево, значит есть цикл, значит можем удалить ребро, входящее только в этот цикл - получили опять не-дерево. Повторяем $m-n+1$ раз, пока не получим дерево.
	\item \textit{В теории было доказано, что любой граф на $n$ вершинах, имеющий меньше $(n-1)$-го ребра, обязательно является несвязным. В случае, когда $|E(G)|\geqslant(n-1)$ граф $G$ может быть как связным, так и несвязным. Сколько ребер должен иметь простой граф на $n$ вершинах, чтобы он гарантированно был связным?}\\
	Докажем такую лемму.
	\newtheorem*{lemma1}{Лемма}
	\begin{lemma1}
	Для любого графа $G$ с $n$ вершинами, $m$ ребрами и $p$ компонентами связности верно следующее неравенство:
	\[
	m \leqslant \dfrac{(n-p)(n-p+1)}{2}
	\]
	\end{lemma1}
	\begin{proof}
	Так как это оценка сверху, то можем считать каждую компоненту связности полным графом ~-- такой простой связный граф содержит максимальное возможное число ребер на фиксированном числе вершин.\\
	Рассмотрим две компоненты связности $C_i$ и $C_j$. И пусть $|V(C_i)|=i$, $|V(C_j)|=j$ и $i \geqslant j > 1$. Заметим, что если заменить б\'{о}льшую компоненту связности на полный граф с $i+1$ вершиной, а меньшую на полный граф с $j-1$ вершинами, то число вершин, очевидно, не изменится. Определим, что будет с числом ребер:
	\[
	\left(\dfrac{i(i+1)}{2}+\dfrac{(j-1)(j+1)}{2}\right)-\left(\frac{i(i-1)}{2}+\frac{j(j-1)}{2}\right) = i-j+1 > 0
	\]
	Число ребер увеличится. Ясно теперь, что максимально возможное число ребер в графе может быть тогда, когда он состоит из $(p-1)$-ой висячей вершины и полного графа из $(n-p+1)$-ой вершины. У такого графа $\dfrac{(n-p)(n-p+1)}{2}$ ребер. Оценка доказана.\qedhere
	\end{proof}
	\newtheorem*{sl_lemma1}{Следствие}
	\begin{sl_lemma1}
	Любой граф $G$ с $n$ вершинами и более чем $\dfrac{(n-2)(n-1)}{2}$ ребрами связен.
	\end{sl_lemma1}
	\begin{proof}
	Это очевидно, так как по лемме у графа с $p=2$ максимальное число ребер $\dfrac{(n-2)(n-1)}{2}$, а у нас их больше.\qedhere
	\end{proof}	
	\item \textit{Доказать, что количество корневых лесов, построенных на $n$ вершинах, равно $(n+1)^{(n-1)}$.}
	Можно комбинаторно, а можно применить интеллект.\\
	Возьмем любой корневой лес.	Введем в этот граф новую вершину, назовем ее "связующей". Соеденим связующую вершину со всеми выделенными корнями всех деревьев в нашем лесу. Очевидно, что получится связный граф, да еще и без циклов. Назовем его "связующее дерево".\\
	Очевидно также, что каждому лесу соответствует ровно одно связующее дерево. И так как все леса были различны, то и все связующие деревья будут различными (ввод новой вершины не может сделать разные графы одинаковыми, иначе будут одинаковые подграфы, а значит исходные деревья были одинаковыми - противоречие).
	Таким образом, мы можем пересчитать все корневые леса через корневые деревья (наше связующее дерево и есть корневое ~-- в нем "отмечена" связующая вершина). А их число мы знаем по теореме Кэли ~-- $(n+1)^{(n-1)}$
	\item \textit{Доказать, что в связном графе два максимальных простых пути имеют общую вершину.}\\
	Предположим обратное: в связном графе есть два максимальных простых пути, которые не имеют ни одной общей вершины:
	\[ x_0,x_1,\ldots,x_n \]
	\[ y_0,y_1,\ldots,y_k \]
	Рассмотрим две вершины $x_n$ и $y_0$: так как граф связный, существует простой путь, соединяющий эти две вершины, причем либо напрямую ребром с одной из вершиной из первого пути, либо с вершинами, которых нет ни в первом, ни во втором пути. Но это значит, что наши максимальные пути не являются максимальными, получили противоречие. Значит, два максимальных простых пути имеют хотя бы одну общую вершину.
\end{enumerate}
\end{document}
